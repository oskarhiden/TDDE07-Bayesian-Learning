\PassOptionsToPackage{unicode=true}{hyperref} % options for packages loaded elsewhere
\PassOptionsToPackage{hyphens}{url}
%
\documentclass[]{article}
\usepackage{lmodern}
\usepackage{amssymb,amsmath}
\usepackage{ifxetex,ifluatex}
\usepackage{fixltx2e} % provides \textsubscript
\ifnum 0\ifxetex 1\fi\ifluatex 1\fi=0 % if pdftex
  \usepackage[T1]{fontenc}
  \usepackage[utf8]{inputenc}
  \usepackage{textcomp} % provides euro and other symbols
\else % if luatex or xelatex
  \usepackage{unicode-math}
  \defaultfontfeatures{Ligatures=TeX,Scale=MatchLowercase}
\fi
% use upquote if available, for straight quotes in verbatim environments
\IfFileExists{upquote.sty}{\usepackage{upquote}}{}
% use microtype if available
\IfFileExists{microtype.sty}{%
\usepackage[]{microtype}
\UseMicrotypeSet[protrusion]{basicmath} % disable protrusion for tt fonts
}{}
\IfFileExists{parskip.sty}{%
\usepackage{parskip}
}{% else
\setlength{\parindent}{0pt}
\setlength{\parskip}{6pt plus 2pt minus 1pt}
}
\usepackage{hyperref}
\hypersetup{
            pdftitle={Lab 2},
            pdfborder={0 0 0},
            breaklinks=true}
\urlstyle{same}  % don't use monospace font for urls
\usepackage[margin=1in]{geometry}
\usepackage{color}
\usepackage{fancyvrb}
\newcommand{\VerbBar}{|}
\newcommand{\VERB}{\Verb[commandchars=\\\{\}]}
\DefineVerbatimEnvironment{Highlighting}{Verbatim}{commandchars=\\\{\}}
% Add ',fontsize=\small' for more characters per line
\usepackage{framed}
\definecolor{shadecolor}{RGB}{248,248,248}
\newenvironment{Shaded}{\begin{snugshade}}{\end{snugshade}}
\newcommand{\AlertTok}[1]{\textcolor[rgb]{0.94,0.16,0.16}{#1}}
\newcommand{\AnnotationTok}[1]{\textcolor[rgb]{0.56,0.35,0.01}{\textbf{\textit{#1}}}}
\newcommand{\AttributeTok}[1]{\textcolor[rgb]{0.77,0.63,0.00}{#1}}
\newcommand{\BaseNTok}[1]{\textcolor[rgb]{0.00,0.00,0.81}{#1}}
\newcommand{\BuiltInTok}[1]{#1}
\newcommand{\CharTok}[1]{\textcolor[rgb]{0.31,0.60,0.02}{#1}}
\newcommand{\CommentTok}[1]{\textcolor[rgb]{0.56,0.35,0.01}{\textit{#1}}}
\newcommand{\CommentVarTok}[1]{\textcolor[rgb]{0.56,0.35,0.01}{\textbf{\textit{#1}}}}
\newcommand{\ConstantTok}[1]{\textcolor[rgb]{0.00,0.00,0.00}{#1}}
\newcommand{\ControlFlowTok}[1]{\textcolor[rgb]{0.13,0.29,0.53}{\textbf{#1}}}
\newcommand{\DataTypeTok}[1]{\textcolor[rgb]{0.13,0.29,0.53}{#1}}
\newcommand{\DecValTok}[1]{\textcolor[rgb]{0.00,0.00,0.81}{#1}}
\newcommand{\DocumentationTok}[1]{\textcolor[rgb]{0.56,0.35,0.01}{\textbf{\textit{#1}}}}
\newcommand{\ErrorTok}[1]{\textcolor[rgb]{0.64,0.00,0.00}{\textbf{#1}}}
\newcommand{\ExtensionTok}[1]{#1}
\newcommand{\FloatTok}[1]{\textcolor[rgb]{0.00,0.00,0.81}{#1}}
\newcommand{\FunctionTok}[1]{\textcolor[rgb]{0.00,0.00,0.00}{#1}}
\newcommand{\ImportTok}[1]{#1}
\newcommand{\InformationTok}[1]{\textcolor[rgb]{0.56,0.35,0.01}{\textbf{\textit{#1}}}}
\newcommand{\KeywordTok}[1]{\textcolor[rgb]{0.13,0.29,0.53}{\textbf{#1}}}
\newcommand{\NormalTok}[1]{#1}
\newcommand{\OperatorTok}[1]{\textcolor[rgb]{0.81,0.36,0.00}{\textbf{#1}}}
\newcommand{\OtherTok}[1]{\textcolor[rgb]{0.56,0.35,0.01}{#1}}
\newcommand{\PreprocessorTok}[1]{\textcolor[rgb]{0.56,0.35,0.01}{\textit{#1}}}
\newcommand{\RegionMarkerTok}[1]{#1}
\newcommand{\SpecialCharTok}[1]{\textcolor[rgb]{0.00,0.00,0.00}{#1}}
\newcommand{\SpecialStringTok}[1]{\textcolor[rgb]{0.31,0.60,0.02}{#1}}
\newcommand{\StringTok}[1]{\textcolor[rgb]{0.31,0.60,0.02}{#1}}
\newcommand{\VariableTok}[1]{\textcolor[rgb]{0.00,0.00,0.00}{#1}}
\newcommand{\VerbatimStringTok}[1]{\textcolor[rgb]{0.31,0.60,0.02}{#1}}
\newcommand{\WarningTok}[1]{\textcolor[rgb]{0.56,0.35,0.01}{\textbf{\textit{#1}}}}
\usepackage{graphicx,grffile}
\makeatletter
\def\maxwidth{\ifdim\Gin@nat@width>\linewidth\linewidth\else\Gin@nat@width\fi}
\def\maxheight{\ifdim\Gin@nat@height>\textheight\textheight\else\Gin@nat@height\fi}
\makeatother
% Scale images if necessary, so that they will not overflow the page
% margins by default, and it is still possible to overwrite the defaults
% using explicit options in \includegraphics[width, height, ...]{}
\setkeys{Gin}{width=\maxwidth,height=\maxheight,keepaspectratio}
\setlength{\emergencystretch}{3em}  % prevent overfull lines
\providecommand{\tightlist}{%
  \setlength{\itemsep}{0pt}\setlength{\parskip}{0pt}}
\setcounter{secnumdepth}{0}
% Redefines (sub)paragraphs to behave more like sections
\ifx\paragraph\undefined\else
\let\oldparagraph\paragraph
\renewcommand{\paragraph}[1]{\oldparagraph{#1}\mbox{}}
\fi
\ifx\subparagraph\undefined\else
\let\oldsubparagraph\subparagraph
\renewcommand{\subparagraph}[1]{\oldsubparagraph{#1}\mbox{}}
\fi

% set default figure placement to htbp
\makeatletter
\def\fps@figure{htbp}
\makeatother


\title{Lab 2}
\author{}
\date{\vspace{-2.5em}}

\begin{document}
\maketitle

\hypertarget{lab-1}{%
\subsection{Lab 1}\label{lab-1}}

Linear and polynomial regression. The dataset TempLinkoping.txt contains
daily average temperatures (in Celcius degrees) at Malmslätt, Linköping
over the course of the year 2018. The response variable is temp and the
covariate is
\[ time = \frac{\mbox{nr of days since begining of year}}{356}\] The
task is to perform a Bayesian analysis of a quadratic regression
\[ temp = \beta_0 + \beta_1 * time + \beta_2 * time^2 + \epsilon, \epsilon \sim \mathcal{N}(0,\,\sigma^{2})\]
\#\#\# a Determining the prior distribution of the model parameters. Use
the conjugate prior for the linear regression model. Your task is to set
the prior hyperparam- eters \(\mu_0, \Omega_0, v_0,\) and
\$\sigma\_0\^{}2 \$ to sensible values. Start with \$\mu\_0 = (-10, 100,
-100)\^{}T, \omega\_0 = 0.01*I\_3, v\_0 = 4 \$ and \(\sigma_0^2 = 1\).
Check if this prior agrees with your prior opinions by simulating draws
from the joint prior of all parameters and for every draw compute the
regression curve. This gives a collection of regression curves, one for
each draw from the prior. Do the collection of curves look rea- sonable?
If not, change the prior hyperparameters until the collection of prior
regression curves agrees with your prior beliefs about the regression
curve. {[}Hint: the R package mvtnorm will be handy. And use your
\(Inv-\chi^2\) simulator from Lab 1.{]}

\begin{Shaded}
\begin{Highlighting}[]
\NormalTok{data =}\StringTok{ }\KeywordTok{read.table}\NormalTok{(}\StringTok{"/Users/oskarhiden/Git/TDDE07 Bayesian Learning/Lab 2/TempLinkoping.txt"}\NormalTok{)}

\NormalTok{time =}\StringTok{ }\KeywordTok{as.numeric}\NormalTok{(}\KeywordTok{as.vector}\NormalTok{(data[}\OperatorTok{-}\DecValTok{1}\NormalTok{,}\DecValTok{1}\NormalTok{]))}
\NormalTok{temp =}\StringTok{ }\KeywordTok{as.numeric}\NormalTok{(}\KeywordTok{as.vector}\NormalTok{(data[}\OperatorTok{-}\DecValTok{1}\NormalTok{,}\DecValTok{2}\NormalTok{]))}

\CommentTok{# 1a)}
\NormalTok{mu_}\DecValTok{0}\NormalTok{ =}\StringTok{ }\KeywordTok{c}\NormalTok{(}\OperatorTok{-}\DecValTok{10}\NormalTok{, }\DecValTok{100}\NormalTok{, }\DecValTok{-100}\NormalTok{)}
\NormalTok{omega_}\DecValTok{0}\NormalTok{ =}\StringTok{ }\FloatTok{0.01}\OperatorTok{*}\KeywordTok{diag}\NormalTok{(}\DecValTok{3}\NormalTok{)}
\NormalTok{omega_}\DecValTok{0}\NormalTok{_inv =}\StringTok{ }\DecValTok{100}\OperatorTok{*}\KeywordTok{diag}\NormalTok{(}\DecValTok{3}\NormalTok{)}
\NormalTok{v_}\DecValTok{0}\NormalTok{ =}\StringTok{ }\DecValTok{4}
\NormalTok{sigma_}\DecValTok{0}\NormalTok{_}\DecValTok{2}\NormalTok{ =}\StringTok{ }\DecValTok{1}

\CommentTok{#draws sigma 2}
\NormalTok{nr_draws =}\StringTok{ }\DecValTok{100}
\NormalTok{x_draws =}\StringTok{ }\KeywordTok{rchisq}\NormalTok{(nr_draws,v_}\DecValTok{0}\NormalTok{)}
\NormalTok{sigma2_draws =}\StringTok{ }\NormalTok{((v_}\DecValTok{0}\NormalTok{)}\OperatorTok{*}\NormalTok{sigma_}\DecValTok{0}\NormalTok{_}\DecValTok{2}\NormalTok{)}\OperatorTok{/}\NormalTok{x_draws}

\CommentTok{#draws beta given simga}
\KeywordTok{library}\NormalTok{(mvtnorm)}
\NormalTok{betas =}\StringTok{ }\KeywordTok{matrix}\NormalTok{(, }\DataTypeTok{nrow=}\NormalTok{nr_draws, }\DataTypeTok{ncol=}\DecValTok{3}\NormalTok{)}
\ControlFlowTok{for}\NormalTok{ (i }\ControlFlowTok{in} \DecValTok{1}\OperatorTok{:}\NormalTok{nr_draws) \{}
\NormalTok{  covar =}\StringTok{ }\NormalTok{sigma2_draws[i]}\OperatorTok{*}\NormalTok{omega_}\DecValTok{0}\NormalTok{_inv}
\NormalTok{  betas[i,] =}\StringTok{ }\KeywordTok{rmvnorm}\NormalTok{(}\DecValTok{1}\NormalTok{, mu_}\DecValTok{0}\NormalTok{, covar)}
\NormalTok{\}}
\NormalTok{A =}\StringTok{ }\KeywordTok{rep}\NormalTok{(}\DecValTok{1}\NormalTok{, }\KeywordTok{length}\NormalTok{(time))}
\NormalTok{B =}\StringTok{ }\NormalTok{time}
\NormalTok{C =}\StringTok{ }\NormalTok{time}\OperatorTok{^}\DecValTok{2}
\NormalTok{x =}\StringTok{ }\KeywordTok{cbind}\NormalTok{(A, B, C) }\CommentTok{#beta order B0, B1, B2}
\NormalTok{y_draws =}\StringTok{ }\NormalTok{betas}\OperatorTok\KeywordTok{t}\NormalTok{(x)}

\CommentTok{#draw predictions}
\KeywordTok{plot}\NormalTok{(time}\OperatorTok{*}\DecValTok{365}\NormalTok{, y_draws[}\DecValTok{1}\NormalTok{,], }\DataTypeTok{type =} \StringTok{"l"}\NormalTok{, }\DataTypeTok{ylim =} \KeywordTok{c}\NormalTok{(}\OperatorTok{-}\DecValTok{60}\NormalTok{,}\DecValTok{70}\NormalTok{))}

\ControlFlowTok{for}\NormalTok{ (i }\ControlFlowTok{in} \DecValTok{2}\OperatorTok{:}\NormalTok{nr_draws) \{}
  \KeywordTok{points}\NormalTok{( y_draws[i,], }\DataTypeTok{type=}\StringTok{"l"}\NormalTok{)}
\NormalTok{\}}
\end{Highlighting}
\end{Shaded}

\includegraphics{Labreport-2_files/figure-latex/1a-1.pdf}

It looks reasonable, since the temperature is higher during summer
(middle of the curves) and lower during winter (start and end of the
curve)

\hypertarget{b}{%
\subsubsection{b}\label{b}}

\begin{Shaded}
\begin{Highlighting}[]
\CommentTok{#1b}
\CommentTok{#posterior}
\KeywordTok{library}\NormalTok{(matlib)}
\end{Highlighting}
\end{Shaded}

\begin{verbatim}
## Warning in rgl.init(initValue, onlyNULL): RGL: unable to open X11 display
\end{verbatim}

\begin{verbatim}
## Warning: 'rgl.init' failed, running with 'rgl.useNULL = TRUE'.
\end{verbatim}

\begin{Shaded}
\begin{Highlighting}[]
\NormalTok{beta_hat =}\StringTok{ }\KeywordTok{inv}\NormalTok{( }\KeywordTok{t}\NormalTok{(x)}\OperatorTok\NormalTok{x ) }\OperatorTok\StringTok{ }\NormalTok{(}\KeywordTok{t}\NormalTok{(x)}\OperatorTok\NormalTok{temp)}
\NormalTok{mu_n =}\StringTok{ }\KeywordTok{inv}\NormalTok{( }\KeywordTok{t}\NormalTok{(x)}\OperatorTok\NormalTok{x}\OperatorTok{+}\NormalTok{omega_}\DecValTok{0}\NormalTok{ )}\OperatorTok\NormalTok{( }\KeywordTok{t}\NormalTok{(x)}\OperatorTok\NormalTok{x}\OperatorTok\NormalTok{beta_hat }\OperatorTok{+}\StringTok{ }\NormalTok{omega_}\DecValTok{0}\OperatorTok\NormalTok{mu_}\DecValTok{0}\NormalTok{)}
\NormalTok{omega_n =}\StringTok{ }\KeywordTok{t}\NormalTok{(x)}\OperatorTok\NormalTok{x}\OperatorTok{+}\NormalTok{omega_}\DecValTok{0}
\NormalTok{v_n =}\StringTok{ }\NormalTok{v_}\DecValTok{0} \OperatorTok{+}\StringTok{ }\KeywordTok{length}\NormalTok{(time)}
\NormalTok{sigma_n_}\DecValTok{2}\NormalTok{ =}\StringTok{ }\NormalTok{(v_}\DecValTok{0}\OperatorTok{*}\NormalTok{sigma_}\DecValTok{0}\NormalTok{_}\DecValTok{2} \OperatorTok{+}\StringTok{ }\NormalTok{(}\KeywordTok{t}\NormalTok{(temp)}\OperatorTok\NormalTok{temp }\OperatorTok{+}\StringTok{ }\KeywordTok{t}\NormalTok{(mu_}\DecValTok{0}\NormalTok{)}\OperatorTok\NormalTok{omega_}\DecValTok{0}\OperatorTok\NormalTok{mu_}\DecValTok{0} \OperatorTok{-}\StringTok{ }\KeywordTok{t}\NormalTok{(mu_n)}\OperatorTok\NormalTok{omega_n}\OperatorTok\NormalTok{mu_n))}\OperatorTok{/}\NormalTok{v_n}

\CommentTok{#draw sigma2}
\NormalTok{nr_draws =}\StringTok{ }\DecValTok{100}
\NormalTok{sigma_draws =}\StringTok{ }\KeywordTok{rchisq}\NormalTok{(nr_draws,v_n) }
\NormalTok{sigma2_draws =}\StringTok{ }\NormalTok{((v_n)}\OperatorTok{*}\NormalTok{sigma_n_}\DecValTok{2}\NormalTok{)}\OperatorTok{/}\NormalTok{sigma_draws}
\end{Highlighting}
\end{Shaded}

\begin{verbatim}
## Warning in ((v_n) * sigma_n_2)/sigma_draws: Recycling array of length 1 in array-vector arithmetic is deprecated.
##   Use c() or as.vector() instead.
\end{verbatim}

\begin{Shaded}
\begin{Highlighting}[]
\NormalTok{betas_post =}\StringTok{ }\KeywordTok{matrix}\NormalTok{(, }\DataTypeTok{nrow=}\NormalTok{nr_draws, }\DataTypeTok{ncol=}\DecValTok{3}\NormalTok{)}
\ControlFlowTok{for}\NormalTok{ (i }\ControlFlowTok{in} \DecValTok{1}\OperatorTok{:}\NormalTok{nr_draws) \{}
\NormalTok{  covar =}\StringTok{ }\NormalTok{sigma2_draws[i]}\OperatorTok{*}\KeywordTok{inv}\NormalTok{(omega_n)}
\NormalTok{  betas_post[i,] =}\StringTok{ }\KeywordTok{rmvnorm}\NormalTok{(}\DecValTok{1}\NormalTok{, mu_n, covar)}
\NormalTok{\}}

\CommentTok{#plot hist of beta and sigma}
\KeywordTok{hist}\NormalTok{(sigma2_draws, }\DataTypeTok{breaks =} \DecValTok{50}\NormalTok{)}
\end{Highlighting}
\end{Shaded}

\includegraphics{Labreport-2_files/figure-latex/1b-1.pdf}

\begin{Shaded}
\begin{Highlighting}[]
\KeywordTok{hist}\NormalTok{(betas[,}\DecValTok{1}\NormalTok{], }\DataTypeTok{breaks =} \DecValTok{50}\NormalTok{)}
\end{Highlighting}
\end{Shaded}

\includegraphics{Labreport-2_files/figure-latex/1b-2.pdf}

\begin{Shaded}
\begin{Highlighting}[]
\KeywordTok{hist}\NormalTok{(betas[,}\DecValTok{2}\NormalTok{], }\DataTypeTok{breaks =} \DecValTok{50}\NormalTok{)}
\end{Highlighting}
\end{Shaded}

\includegraphics{Labreport-2_files/figure-latex/1b-3.pdf}

\begin{Shaded}
\begin{Highlighting}[]
\KeywordTok{hist}\NormalTok{(betas[,}\DecValTok{3}\NormalTok{], }\DataTypeTok{breaks =} \DecValTok{50}\NormalTok{)}
\end{Highlighting}
\end{Shaded}

\includegraphics{Labreport-2_files/figure-latex/1b-4.pdf}

\begin{Shaded}
\begin{Highlighting}[]
\CommentTok{# calculate y, predicted temp for all draws}
\NormalTok{y_post =}\StringTok{ }\NormalTok{betas_post}\OperatorTok\KeywordTok{t}\NormalTok{(x) }

\CommentTok{#Calculate mean for y each day form all predictions}
\NormalTok{y_median =}\StringTok{ }\KeywordTok{apply}\NormalTok{(y_post,}\DecValTok{2}\NormalTok{,median)}
\KeywordTok{plot}\NormalTok{(temp, }\DataTypeTok{col=}\StringTok{"blue"}\NormalTok{)}
\KeywordTok{points}\NormalTok{(time}\OperatorTok{*}\DecValTok{365}\NormalTok{, y_median, }\DataTypeTok{type=}\StringTok{"l"}\NormalTok{)}

\CommentTok{#approx curves for 95% equal tail by sorting}
\CommentTok{#sorted_y_post = apply(y_post,2,sort,decreasing=F)}
\CommentTok{#points(sorted_y_post[round(nr_draws*0.025),], type="l", col="red")}
\CommentTok{#points(sorted_y_post[round(nr_draws*0.975),], type="l", col="red")}

\NormalTok{quantile_y_post =}\StringTok{ }\KeywordTok{apply}\NormalTok{(y_post,}\DecValTok{2}\NormalTok{,quantile, }\DataTypeTok{probs=}\KeywordTok{c}\NormalTok{(}\FloatTok{0.025}\NormalTok{,}\FloatTok{0.975}\NormalTok{))}
\KeywordTok{points}\NormalTok{(quantile_y_post[}\DecValTok{1}\NormalTok{,], }\DataTypeTok{col=}\StringTok{"green"}\NormalTok{, }\DataTypeTok{type=}\StringTok{"l"}\NormalTok{)}
\KeywordTok{points}\NormalTok{(quantile_y_post[}\DecValTok{2}\NormalTok{,], }\DataTypeTok{col=}\StringTok{"green"}\NormalTok{, }\DataTypeTok{type=}\StringTok{"l"}\NormalTok{)}
\end{Highlighting}
\end{Shaded}

\includegraphics{Labreport-2_files/figure-latex/1b-5.pdf}

The posterior probability intevall shows us were our real model would be
wthin with 95\% credability (from beta). It does not show were the
actual y is by 95\% credability.

\hypertarget{c}{%
\subsubsection{c}\label{c}}

\begin{Shaded}
\begin{Highlighting}[]
\NormalTok{x_hat =}\StringTok{ }\KeywordTok{which.max}\NormalTok{(y_median)}
\NormalTok{y_median[x_hat]}
\end{Highlighting}
\end{Shaded}

\begin{verbatim}
## [1] 16.1585
\end{verbatim}

\begin{Shaded}
\begin{Highlighting}[]
\CommentTok{#From previous data}
\NormalTok{all_hot_days =}\StringTok{ }\KeywordTok{apply}\NormalTok{(y_post, }\DecValTok{1}\NormalTok{, which.max)}
\KeywordTok{hist}\NormalTok{(all_hot_days, }\DataTypeTok{breaks=}\DecValTok{10}\NormalTok{)}
\end{Highlighting}
\end{Shaded}

\includegraphics{Labreport-2_files/figure-latex/1c-1.pdf}

\begin{Shaded}
\begin{Highlighting}[]
\CommentTok{#from derivation}
\NormalTok{all_hot_days2 =}\StringTok{ }\NormalTok{(}\OperatorTok{-}\NormalTok{betas_post[,}\DecValTok{2}\NormalTok{]}\OperatorTok{/}\NormalTok{(}\DecValTok{2}\OperatorTok{*}\NormalTok{betas_post[,}\DecValTok{3}\NormalTok{]))}\OperatorTok{*}\DecValTok{365}
\KeywordTok{hist}\NormalTok{(all_hot_days2)}
\end{Highlighting}
\end{Shaded}

\includegraphics{Labreport-2_files/figure-latex/1c-2.pdf}

\hypertarget{d}{%
\subsubsection{d}\label{d}}

We will have a laplace distribution for the betas with u0=0 and
omega0=I(8)*lambda. Were lambda is the smoothing coefficient that
decides how much the betas are allowed to differ from zero and u0 sets
the mean of these to be zero in the beginning. This will make a few
betas go into the fat tails of the laplace distribution and a few to end
up at 0, the prior mean.

\end{document}
